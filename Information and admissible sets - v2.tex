%% Page settings.
\documentclass[10pt,a4paper,twoside]{article}
\usepackage[top=1.2in, left=1.2in, bottom=1.2in, right=1.2in]{geometry}
%==================================================%
%% Encoding packages.
\usepackage[UKenglish]{babel}
\usepackage[nodayofweek]{datetime}
\usepackage[T1]{fontenc}
\usepackage[utf8]{inputenc}
\usepackage{amsmath}
\usepackage{amsthm}
\DeclareMathAlphabet\mathbb{U}{msb}{m}{n}
%\usepackage{amsfonts}
\usepackage{amssymb}
\usepackage{calc}
\usepackage{natbib}
\usepackage{color}
\usepackage{subcaption}
%==================================================%
%% Document details.
\usepackage{titling}
\title{Information and admissible sets}
\author{Jeff Rowley}
\newcommand{\thedate}{\today}
% Enter document details here.
\newcommand{\details}{C:/Dropbox/TeXTemplates/}
% Enter the file path here for the UCL logo and bibliography.
% Change this file path for different computer systems.
\newcommand{\homePC}{C:/Users/Jeffro/Documents/}
\newcommand{\workPC}{U:/}
%==================================================%
%% Date macro.
\newcommand{\theseason}[1]{
\ifcase \month 
\or Winter\or Winter\or Spring\or Spring\or Spring\or Summer\or Summer\or Summer\or Autumn\or Autumn\or Autumn\or Winter\fi}
% Displays the season.
% Obsolete in this template.
%==================================================%
%% Useful packages.
\usepackage{enumerate}		% Lists.
\usepackage{bbm}			% Indicator functions.
\usepackage{lipsum}			% Random text generator.
\usepackage{MnSymbol}		% Arrows.
\usepackage{graphicx}		% Graphics.
\usepackage{mathtools}		% Math*lap command.
%==================================================%
%% Theorem environments.
\newcounter{countthm}[section]
\newcounter{countlem}[section]
\newcounter{countex}[section]
% Creates new counters which reset at each new section. 
\renewcommand{\thecountthm}{\thesection.\arabic{countthm}}
\renewcommand{\thecountlem}{\thesection.\arabic{countlem}}
\renewcommand{\thecountex}{\thesection.\arabic{countex}}
% Redefines counters to include the section number.
\newtheorem{thm}[countthm]{Theorem}
% Creates a theorem environment - type \thm to begin.
\newtheorem{lem}[countlem]{Lemma}
% Creates a lemma environment - type \lem to begin.
\newtheorem{ex}[countex]{Example}
% Creates an example environment - type \ex to begin.
\newtheorem*{Acknowledgements}{Acknowledgements}
% Creates a thanks environment.
\newcommand{\newthm}[1]{\newtheorem*{#1}{#1}}
% A macro that makes defining new theorem environments quick.
% Type \newthm{<Theorem name here>} to begin.
%==================================================%
%% Math operators.
\DeclareMathOperator*{\plim}{plim}
% Writes plim in math environment - type \plim to enter.
\DeclareMathOperator*{\argmax}{argmax}
% Writes argmax in math environment - type \argmax to enter.
\DeclareMathOperator*{\argmin}{argmin}
% Writes argmin in math environment - type \argmin to enter.
\DeclareMathOperator*{\argsup}{argsup}
% Writes argsup in math environment - type \argsup to enter.
\DeclareMathOperator*{\arginf}{arginf}
% Writes arginf in math environment - type \arginf to enter.
\newcommand\independent{\protect\mathpalette{\protect\independenT}{\perp}} 
\def\independenT#1#2{\mathrel{\rlap{$#1#2$}\mkern2mu{#1#2}}} 
% Pastes an independence symbol - type \independent to enter.
\DeclareMathOperator*{\generates}{:.}
% Write generate symbol (identification).
\DeclareMathOperator*{\generated}{.:}
% Write generated symbol (identification).
%==================================================%
%% Abbreviations.
\newcommand{\US}{United States}
% Use \US macro when use is as a noun.
% Use U.S. when use is as an adjective.
%==================================================%
%% Equation numbering.
\numberwithin{equation}{section}
% Numbers equations up to the subsection. To change level,
% replace subsection with section.
%==================================================%
%% Counters.
\newcounter{saveenumi}
\newcounter{saveenumi1}
\setcounter{section}{0}
%==================================================%
\newcommand{\ESRC}{I gratefully acknowledge financial support from the Economic and Social Research Council (ESRC).}
\newcommand{\RLaTeX}{I acknowledge the \emph{R} and \emph{\LaTeX} communities, and the wealth of knowledge that they have made freely available to all.}
%==================================================%
%% The title page.
\makeatletter				
% Changes @ to catcode 11.
\renewcommand{\@maketitle}{
\null
\graphicspath{ {\details} }
\flushleft{\includegraphics[width=40mm]{UCL_Logo_Orange}}
\hspace{5mm}
\normalsize Department of Economics, University College London\\
\vskip\bigskipamount
\leaders\vrule width \textwidth\vskip0.4pt 
\vskip\bigskipamount 
\nointerlineskip
% This completes the UCL banner.
\begin{center}
\begin{minipage}{100mm}
\begin{center}
\vspace{20mm}
\LARGE
\textbf{
\@title}
\par
\vspace{10mm}
\normalsize
\@author
\par
\vspace{5mm}
\normalsize
\thedate
\end{center}
\end{minipage}
\end{center}
}
\makeatother				
% Reverts @ to catcode 12.
%==================================================%
%% Packages to load at end of preamble. 
% Note conflict between Tkz-euclide package set and 
% game theory package set. Load one or the other.
\usepackage{hyperref}
%\usepackage[numbered]{mcode}

%% Tkz-euclide package set.
%\usepackage{tkz-euclide}
%\usepackage{pgfplots}
%\usepgfplotslibrary{external} 
%\tikzexternalize[prefix=tikz/]

%% Game theory package set.
\usepackage{pstricks}	
\usepackage{egameps}		 
\usepackage{pst-3d}			
\usepackage{sgame}	
\renewcommand{\gamestretch}{1.5}
%==================================================%
%% Further notes regarding egameps package.
% The egameps package is incompatible with this template
% due to UCL logo. Solution is to independently run 
% egameps through on a latex blank document, then insert 
% pdf into this document. Recall that to run egameps:
% "latex" --> "DVi->PS" --> "PS->PDF"
% then use "includegraphics[]" with trim option. 
%==================================================%
%% Headers and footers.
\usepackage{fancyhdr}
\pagestyle{fancy}
\renewcommand{\sectionmark}[1]{\markright{\thesection.\ #1}}
% This redefines the \rightmark command so that the section number does not appear.
% NOTE: To remove the section number, delete <<\thesection.\>>
\lhead[\thepage]{\rightmark}
\rhead[\rightmark]{\thepage}
\chead[]{}
\cfoot[]{}
%\lfoot[\thetitle]{}
%\rfoot[]{\theauthor}
\renewcommand*\thesection{\arabic{section}}
\usepackage{epigraph}
%==================================================%
%% Start of document.
\begin{document}
\maketitle
\vspace{10mm}
\begin{abstract}
\noindent <<Abstract here>>
\begin{Acknowledgements}
{\RLaTeX} I thank Andrew Chesher and Toru Kitagawa for their supervision and support. I further thank Adam Rosen for helpful discussion. {\ESRC}
\end{Acknowledgements}
\end{abstract}
\vspace{5mm}
%==================================================%
%% Document.
%==================================================%
I explore the effect of incorporating information for a non-parametric binary choice model. The model permits endogenous variation in a scalar random variable that is due to non-random selection, and it is the average causal effect of this endogenous variable on the outcome variable that is of interest. The model embeds an exclusion restriction and an independence restriction that together define an instrumental variable but is silent as to the relationship between the endogenous variable and the instrumental variable. I restrict the relationship between the outcome variable and the endogenous variable up to a non-parametric threshold crossing function. The model is credible \citep{book.manski} in that it embeds only weak non-verifiable restrictions, but does not identify the average causal effect of the endogenous variable on the outcome variable.\footnote{Assumptions that cannot be tested using data. The model does embed some non-trivial non-verifiable restrictions that might be relaxed.} Rather, the model partially identifies the average causal effect of the endogenous variable on the outcome variable. 

I define information to be those additional characteristics of economic agents that are observable with the caveat that these characteristics be exogenous and relevant to the latent structure. It is convenient to think of such characteristics as being predetermined and immutable; characteristics that result from choices that are made jointly with the outcome variable are excluded by the definition. Accordingly, exogenous variables and instrumental variables are each regarded as information, and I distinguish between these classes of information. I study how the admissible set of values for the average causal effect of the endogenous variable on the outcome variable changes as each class of information is incorporated into the model separately. 

It is useful to distinguish between classes of information since each class enters the latent structure in a different way. Exogenous variables are permitted to enter the structural equation for the outcome variable and to determine the endogenous variable. As such, exogenous variables can be seen to enrich both individual response and individual selection, respectively. An important consequence is that the causal effect of the endogenous variable on the outcome variable depends upon the value of the exogenous variables when individual response is enriched. In contrast, instrumental variables are excluded from the structural equation for the outcome variable by definition and so only enrich individual selection. Given this, the effect of incorporating information is different depending upon the class of information that is being incorporated into the model.  

Incorporating information of either class is generally sensible for a number of reasons. Firstly, incorporating information is known to be efficient; variation that is attributable to an observable variable is instead attributable to unobservable heterogeneity when that variable is omitted. Secondly, the effect of incorporating information for partially identifying models is not well-documented; one hypothesis is that incorporating information narrows bounds on admissible sets. Such an effect is not documented in identifying models precisely because such models deliver a point estimate (a set of length zero), but point estimates may shift as information is incorporated. A contribution that I make is in showing that \color{red} incorporating information leads to narrower bounds on the admissible sets \color{black} that are delivered by the model. A further reason to particularly favour incorporating exogenous variables is that the average causal effect of the endogenous variable on the outcome variable in identifiable sub-populations can be recovered. I name this structural characteristic the conditional average causal effect of the endogenous variable on the outcome variable, and index it by the conditioning value.\footnote{The conditioning value is specifically the value of the exogenous variables. \cite{hEvY05} defines a parameter $ATE(x)$ that is equivalent to the conditional average causal effect of the endogenous variable on the outcome variable at the conditioning value $x$. \cite{kHt10} and \cite{13.misc.abrevaya} instead refer to this parameter as the conditional average treatment effect and abbreviate this to $CATE(x)$.} Understanding the effect of an intervention in sub-populations can be interesting if the intervention can be targeted or if the intervention is to be applied elsewhere in a population that differs according to its observable characteristics. 

A relevant question is how to relate conditional causal effects to (unconditional) causal effects. More precisely, how does the average causal effect of the endogenous variable on the outcome variable relate to its conditional counterparts? I show that the average causal effect of the endogenous variable on the outcome variable can be expressed as a Minkowski summation of its conditional counterparts when the non-parametric binary choice model is augmented. I derive sharp bounds on the conditional average causal effect by applying random set theory. I employ the capacity (or containment) functional as in \cite{crs13} as a matter of choice, rather than the Aumann expectation as in \cite{bEmOImOF12}. As I show that the average causal effect of the endogenous variable on the outcome variable can be expressed as a Minkowski summation of its conditional counterparts, I derive sharp bounds on the average causal effect of the endogenous variable on the outcome variable.\footnote{\cite{book.molchanov} is a useful companion in the study of random sets.} I establish the conditions under which bounds on conditional causal effects can be informative about bounds on (unconditional) causal effects. That is, I establish the conditions under which bounds on conditional causal effects can be used to narrow bounds on (unconditional) causal effects, exploiting the mapping from one to the other.      

I demonstrate application of the non-parametric binary choice model, elucidating the practical difficulties that arise when estimating set identifying models (focusing on those issues that arise from incorporating information). As in \cite{cr13}, I estimate the average causal effect of childbirth on a mother's labour force participation using US census data. I extend \cite{cr13} in a number of ways. Firstly, I report statistical uncertainty in the estimate of the average causal effect of childbirth on a mother's labour force participation using a method that is outlined in \cite{cHlr13}. Secondly, I enrich the support of the instrumental variable and explore the effect that this has on the admissible set of values for the average causal effect of childbirth on a mother's labour force participation, and on its accompanying confidence region. Thirdly, I enrich individual response by permitting the structural equation for labour force participation to depend upon the age and other such predetermined and immutable characteristics of mothers. I discuss the complication of calculating statistical uncertainty when exogenous variables are permitted to enter the structural equation for labour force participation. With respect to the second and third extensions, it is necessary that I augment the model by embedding additional restrictions. In fact, \cite{cr13} describe the augmented non-parametric binary choice model that I assume but simplify this model for application (by excluding exogenous variables from the structural equation for the outcome variable). I discuss how the augmented model relates to the simplified model in each case and the credibility of the additional restrictions that are embedded in the augmented model.
%==================================================%
\section*{Related research}
Other notable non-parametric binary choice models are described in \cite{bp97} and \cite{sHvY11}, and general non-parametric models of choice are described in \cite{c05}, \cite{kI09} and \cite{c10}.

\cite{bp97} assumes a triangular model (the model embeds a structural equation for the outcome variable and a structural equation for the endogenous variable; see \cite{sTw60} for a detailed discussion of triangular models) that relaxes separability of unobservable heterogeneity in the structural equation for the outcome variable. The cost is that the model is no longer silent as to the relationship between the endogenous variable and the instrumental variable. The model does not permit exogenous variables to enter the structural equation for the outcome variable. I discuss the credibility of separability of unobservable heterogeneity in the main text. \cite{sHvY11} assumes a triangular model but maintains separability of unobservable heterogeneity in the structural equation for the outcome variable. The model permits exogenous variables to enter the structural equation for the outcome variable.

\cite{c05} and \cite{kI09} describe non-parametric models that permit continuous variation in the outcome variable. \cite{c05} assumes a triangular model that relaxes separability of unobservable heterogeneity in the structural equation for the outcome variable. The model permits exogenous variables to enter the structural equation for the outcome variable, although local invariance of the structural equation for the outcome variable to variation in the exogenous variables is embedded. The model is uninformative when there is binary variation in the endogenous variable but is informative when there is discrete variation. \cite{kI09} extends \cite{bp97} to permit discrete and continuous variation in the outcome variable, and studies commonly invoked restrictions on covariation of the instrumental variable and unobservable heterogeneity.

\cite{c10} describes an ordered choice model that permits discrete variation in the outcome variable. \cite{c10} assumes a single equation model that relaxes separability of unobservable heterogeneity in the structural equation for the outcome variable, although monotonicity of the structural equation for the outcome variable in unobservable heterogeneity is embedded. The model permits binary or discrete variation in the endogenous variable.
%==================================================%
\section*{Notation}
I study a probability space $(\Omega,\Sigma,\mathbb{P})$. I define random variables on this probability space. I write random variables as upper case Latin letters, and I write realisations (or specific values) of random variables as lower case Latin letters. I write the support of $A$ as $\mathcal{R}_A$. I write the counterfactual value of $A$ when $B$ has a causal interpretation and is externally fixed as $A(b)$. I write the average causal effect of $B$ on $A$ as $ACE(B\rightarrow A)$, and the conditional average causal effect of $B$ on $A$ given $C$ as $ACE(B\rightarrow A|c)$.

I refer to $Y$ as the outcome variable, to $D$ as the endogenous variable, to $X$ as the endogenous variable, to $Z$ as the instrumental variable, and to $U$ as unobservable heterogeneity. Despite the use of \emph{the}, I permit $(X,Z)$ to be vectors. I write the structural equation for the outcome variable as $h$, and the structural equation for the endogenous variable as $g$.  

I write the expectation operator as $\mathbb{E}$, and the indicator function as $\mathbbm{1}$. I write $A$ is independent of $B$ as $A\independent B$. To distinguish between population and sample quantities, I subscript sample quantities by $n$. 

I introduce further terminology and notation in Figure~\ref{fig:models} through Figure~\ref{fig:partials}. This specifically relates to models and structures, and is consistent with the approach that is formally laid out in \cite{h50} and in \cite{krE50}. 
%==================================================%
\section*{Application}
I estimate the average causal effect of childbirth on a mother's labour force participation using {\US} census data. The data are obtainable from \cite{Angristdatabank}, and are described in \cite{ae98}. To summarise, the dataset consists of 254,654 households that were recorded as part of the 1980 {\US} census. The dataset specifically contains observations of married households with at least two children under the age of 18 years and where the mother is aged between 21 years and 35 years. For clarity, I translate each variable in the data into the mathematical notation that I employ.
\begin{align*}
Y&\equiv\mathbbm{1}[\text{Mother is employed in 1979}]\\
D&\equiv\mathbbm{1}[\text{Three or more children in the household}]
\end{align*}
As $(X,Z)$ are continually redefined in the main text, I do not define these variables as I do $(Y,D)$. Instead, I note those variables that at some point or other form part of the definition of $(X,Z)$. $X$ is a function of mother's race or ethnicity (shortened to race).\footnote{\cite{ae98} also treats mother's age, and mother's age at the time of her first birth (shortened to birthing age) as exogenous variables.} $Z$ is a function of whether the oldest two children in the household share the same gender, and whether the second pregnancy was a multiple pregnancy.

I refer to the application throughout the main text so as to illustrate how technical conditions on variables and on the relationship between variables restrict the behaviour of economic agents, in this case mothers. 
%==================================================%
\section{A non-parametric model of binary choice}
I begin by introducing the non-parametric binary choice model that is described in \cite{cr13}. The model constitutes the set of structures that are consistent with Restriction \small{M}\normalsize{1} through Restriction \small{M}\normalsize{4}. 
\newthm{Axiom}
\begin{Axiom}
Economic agents are utility maximising, selecting between alternatives in a choice set according to the utility that they attach to that choice. Utility is perfectly observable by economic agents and is determined by a well-defined utility function for each choice. Each agent is permitted to value each choice differently. 
\end{Axiom}
\begin{enumerate}[\bf \small{M}\normalsize{1}.] 
\item $\mathrlap{\text{\bf Discrete support.}}$\phantom{\textbf{Joint independence.}} $(Y,D,X,Z)$ are observable and have discrete supports (with at least two points of support). Further, $(Y,D)$ have binary supports and are normalised such that\begin{enumerate}[(a.)]
\item $\mathcal{R}_Y=\lbrace 0,1\rbrace$ and
\item $\mathcal{R}_D=\lbrace 0,1\rbrace$,
\end{enumerate}
respectively.
\end{enumerate}
Restriction \small{M}\normalsize{1} is a verifiable restriction. $(Y,D,X,Z)$ are observable and so it is trivial to verify that each variable satisfies its support restriction. $(Y,D)$ are normalised to be consistent with the application, but any other supports $\lbrace y_0,y_1\rbrace$ and $\lbrace d_0,d_1\rbrace$ can be generated by an affine transformation of $h$ and of $g$. 
\begin{enumerate}[\bf \small{M}\normalsize{2}.] 
\setcounter{enumi}{2}
\item $\mathrlap{\text{\bf Scalar $\boldsymbol{U}$.}}$\phantom{\textbf{Joint independence.}} $U$ is an unobservable scalar such that $\mathcal{R}_U$ is an open subset of $\mathbb{R}$ with strictly positive Lebesgue measure.
\end{enumerate}
Suppose that the utility of choice $m$ is determined by the utility function
\[V_m(O,N)\]
where $O$ is the vector of all observable determinants of utility, and $N$ is the vector of all unobservable determinants. Then if the cardinality of $\text{supp}$ is no greater than the cardinality of $\mathbb{R}$ then there exists a bijection $\text{supp}(N)\mapsto \mathbb{R}$, which has output $U$. It follows that there exists a utility function $\tilde{V}_m$ such that
\[V_m(O,N)\equiv\tilde{V}_m(O,U).\]
Restriction \small{M}\normalsize{2} can then be regarded as a normalisation rather than as a restriction in that $U$ is simply a projection of a high-dimensional vector onto $\mathbb{R}$ that preserves uniqueness. 
\begin{enumerate}[\bf \small{M}\normalsize{3}.] 
\setcounter{enumi}{2}
\item $\mathrlap{\text{\bf Joint independence.}}$\phantom{\textbf{Joint independence.}} $U\independent (X,Z)$.
\end{enumerate}
Restriction \small{M}\normalsize{2} is a non-verifiable restriction. The restriction is stronger than the marginal independence relations $U\independent X$ and $U\independent Z$ since it imposes extra constraints on the joint distribution $\mathbb{P}(u,x,z)$. 
%
%
%
%
%
%is a non-verifiable restriction. The restriction implies the marginal independence relations 
%\begin{equation}
%U\independent X\label{M2:1}
%\end{equation}
%and
%\begin{equation}
%U\independent Z,\label{M2:2}
%\end{equation}
%and restricts the dependence between $U$ and combinations of $(X,Z)$. Importantly, the restriction does not rule out dependence between $X$ and $Z$. In the context of the application, ability is independent of birthing age and the occurrence of a multiple second birth, but multiple second births are .    
%
%
%as Restriction \small{M}\normalsize{2} implies (\ref{M2:1}) and (\ref{M2:2}), but also restricts the joint distribution of $(X,Z)$. 
%independence (of $U$ and $X$, and of $U$ and $Z$) in that it 
%
%The restriction is stronger than marginal independence (of $U$ and $X$, and of $U$ and $Z$); in fact, Restriction \small{M}\normalsize{2} implies marginal independence. The distinction between 
%
%
%The restriction states that labour market ability is independent of a mother's race or ethnicity, and is independent of whether her second pregnancy was a multiple pregnancy. 
%\begin{enumerate}[\bf \small{M}\normalsize{3}.] 
%\setcounter{enumi}{2}
%\item $\mathrlap{\text{\bf Exclusion.}}$\phantom{\textbf{Joint independence.}} 
%\end{enumerate}
%\begin{enumerate}[\bf \small{M}\normalsize{4}.] 
%\setcounter{enumi}{3}
%\item $\mathrlap{\text{\bf Monotonicity.}}$\phantom{\textbf{Joint independence.}} Restriction IV.
%\end{enumerate}
%\begin{enumerate}[\bf I.]
%\item \textbf{Discrete support.} $(Y,D)$ have binary support. $(X,Z)$ have discrete support.
%\item \textbf{Independence.} $U\independent (X,Z)$.
%\item \textbf{Exclusion.} $Y=h(D,X,U)$.
%\item \textbf{Monotonicity.} $h(D,X,U)=\mathbbm{1}[p(D,X)-U>0]$.
%\end{enumerate}
%Normalisation on $U$ to be distributed uniformly on $[0,1]$
%
%Independence and Exclusion are the normal rank and order conditions. Exclusion states that $Z$ does not affect $Y$ directly. If a mother has the same gender children then this should not affect her decision to participate or not. Similarly, if mother has twins or higher plurality birth then this should not directly affect her decision to participate. Although having two young children versus two older children may differentially affect decision to go to work...
%
%Monotonicity states that the threshold is independent of $U$. This means that all mothers must never increase their labour supply or never decrease their labour supply. It rules out some mothers increasing their labour supply in response to children, and others decreasing. Non-monotonic response is considered in \cite{bp97}.
%==================================================%
\section{Credibility in economic modelling}
%==================================================%
\section{Incorporating information}
\subsection{Enriching the support of the instrumental variable}
\subsection{Enriching individual response}
%==================================================%
\section{A non-parametric model of binary choice}
I begin by introducing the non-parametric binary choice model that is studied in \cite{cr13}. I discuss the credibility of the restrictions that 
are embedded in the model and the limitations of these restrictions. I then augment the model and incorporate information. Firstly, I enrich the support of the instrumental variable. Secondly, I enrich the structural function for the outcome variable by allowing individual response to depend upon exogenous variables as well as on the endogenous variable. In both cases, I discuss the credibility of the restrictions that are embedded in the augmented model and the limitations of these restrictions. I discuss the effect of incorporating both classes of information on the admissible set of values for the average causal effect of the endogenous variable on the outcome variable. 
\subsection{A baseline model}
\cite{cr13} 
\begin{align*}
Y&=\mathbbm{1}[p(D)>U]\\
U&\independent Z
\end{align*}
%==================================================%
%% Diagrams
%==================================================%
\begin{figure}[p]
\centering
\begin{subfigure}{0.8\textwidth}
  \centering
  \includegraphics[width=\textwidth]{U:/GitHub/Covariatesets/Diagrams/Identification.pdf}
  \caption{A model $\mathcal{M}$ is a set of structures that forms a proper subset of the class of all structures $\mathcal{S}$. Each structure in $\mathcal{M}$ generates a probability distribution in the class of all probability distributions (of observable variables) $\mathcal{P}$. Then the image $\mathcal{I}$ is the set of all probability distributions that are generated by structures in $\mathcal{M}$.}
  \label{fig:model}
  \end{subfigure}
\begin{subfigure}{0.8\textwidth}
  \centering
  \includegraphics[width=\textwidth]{U:/GitHub/Covariatesets/Diagrams/Observationalrestrictiveness.pdf}
  \caption{A structure $S$ is incompatible with data if it generates a probability distribution (of observable variables) $P$ that is distinct from a realised probability distribution $\mathbb{P}$. If all structures in $\mathcal{M}$ are incompatible with data then $\mathcal{M}$ is said to be observationally restrictive, and is falsified. This condition is equivalent to $\mathbb{P}\in\mathcal{P}\setminus\mathcal{I}$.}
  \label{fig:obs.restrict}
  \end{subfigure}
\caption{Structures, models, probability distributions (of observable variables), and falsifiability.}
\label{fig:models}
\end{figure}
%==================================================%
\begin{figure}[p]
\centering
\begin{subfigure}{0.8\textwidth}
  \centering
  \includegraphics[width=\textwidth]{U:/GitHub/Covariatesets/Diagrams/Pointidentification.pdf}
  \caption{A model $\mathcal{M}$ is said to identify a structure $S$ if the probability distribution (of observable variables) $P$ that is generated by $S$ is distinct from those generated by other structures in $\mathcal{M}$. The structures $S_a$, $S_b$ and $S_c$ are said to be observationally equivalent as they all generate $P$ but $S_b$ and $S_c$ are not admitted by $\mathcal{M}$. As $S_a$ is the only structure that is admitted by $\mathcal{M}$ and that generates $P$, $S_a$ is identified by $\mathcal{M}$. For completeness, $\mathcal{M}$ is said to be uniformly identifying if it identifies each structure that it admits.}
  \label{fig:identify}
  \end{subfigure}
  \begin{subfigure}{0.8\textwidth}
  \centering
  \includegraphics[width=\textwidth]{U:/GitHub/Covariatesets/Diagrams/Setidentification.pdf}
  \caption{As $S_a$ and $S_b$ are observationally equivalent and are both admitted by $\mathcal{M}$ then $\mathcal{M}$ does not identify either $S_a$ or $S_b$. Nonetheless, as $\mathcal{M}$ restricts the set of observationally equivalent structures that generate $P$ to $S_a$ and $S_b$ then $\mathcal{M}$ partially identifies $S_a$ (and $S_b$ to within $\lbrace S_a,S_b\rbrace$).}
  \label{fig:partial}
  \end{subfigure}
\caption{Identification and non-identification of a structure, and partial identification of a structure.}
\label{fig:identification}
\end{figure}
%==================================================%
\begin{figure}[p]
\centering
\begin{subfigure}{0.8\textwidth}
  \centering
  \includegraphics[width=\textwidth]{U:/GitHub/Covariatesets/Diagrams/Characteristic.pdf}
  \caption{A structural characteristic $\chi$ is a function of a structure $S$. A model $\mathcal{M}$ can be partitioned such that structures in a partition deliver the same value for $\chi$. Structures in the red partition $\color{red}\mathcal{M}$ deliver the value $a$ for $\chi$, and structures in the red partition $\color{blue}\mathcal{M}$ deliver the value $b$ for $\chi$. If $\chi$ is constant across all observationally equivalent structures that $\mathcal{M}$ admits then $\mathcal{M}$ is said to identify $\chi$. As $\chi(S_a)$ is equal to $\chi(S_b)$ (is equal to $a$) $\mathcal{M}$ identifies $\chi$.}
  \label{fig:characteristic}
  \end{subfigure}
  \begin{subfigure}{0.8\textwidth}
  \centering
  \includegraphics[width=\textwidth]{U:/GitHub/Covariatesets/Diagrams/Uniform.pdf}
  \caption{If $\mathcal{M}$ identifies $\chi$ for all structures in $\mathcal{M}$ then $\mathcal{M}$ is said to uniformly identify $\chi$. The class of all probability distributions (of observable variables) is partitioned into the blue partition $\color{red}\mathcal{P}$ and into the red partition $\color{blue}\mathcal{P}$. Probability distributions in $\color{red}\mathcal{P}$ are generated by (potentially many) structures in $\color{red}\mathcal{M}$, and probability distributions in $\color{blue}\mathcal{P}$ are generated by (potentially many) structures in $\color{blue}\mathcal{M}$. It is important that the number of partitions in $\mathcal{M}$ and in $\mathcal{P}$ are equal, although that number can be countably infinite. In the context of Figure~\ref{fig:uniform} $\mathcal{M}$ uniformly identifies $\chi$ since observationally equivalent structures that $\mathcal{M}$ admits are in the same colour of $\mathcal{M}$. More conveniently, whether $\mathcal{M}$ uniformly identifies $\chi$ can be determined by the existence of an identifying correspondence $G$, a functional. $\color{red}P$ is a probability distribution in $\color{red}\mathcal{P}$, and $\color{blue}P$ is a probability distribution in $\color{blue}\mathcal{P}$. Then $\mathcal{M}$ uniformly identifies $\chi$ if the value of $G(\color{red}P\color{black})$ is $a$ and if the value of $G(\color{blue}P\color{black})$ is $b$, holding for any such $\color{red}P$ and $\color{blue}P$. Notice that if $\mathcal{M}$ uniformly identifies all $\chi$ then $\mathcal{M}$ also uniformly identifies structures.}
  \label{fig:uniform}
  \end{subfigure}
  \caption{The identification of structural characteristics, and identifying correspondences.}
  \label{fig:characteristics}
\end{figure}
%==================================================%
\begin{figure}[p]
\centering
\begin{subfigure}{0.8\textwidth}
\centering
\caption{A structural characteristic $\chi$ is a function of a structure $S$. A model $\mathcal{M}$ can be partitioned such that structures in a partition deliver the same value for $\chi$. Structures in the red partition $\color{red}\mathcal{M}$ deliver the value $a$ for $\chi$, structures in the blue partition $\color{blue}\mathcal{M}$ deliver the value $b$ for $\chi$, and structures in the yellow partition $\color{yellow}\mathcal{M}$ deliver the value $c$ for $\chi$. The class of all probability distributions (of observable variables) $\mathcal{P}$ is partitioned into the red partition $\color{red}\mathcal{P}$, into the blue partition $\color{blue}\mathcal{P}$, into the yellow partition $\color{yellow}\mathcal{P}$ and into the grey partition $\color{gray}\mathcal{P}$. Probability distributions in a colour of $\mathcal{P}$ are generated by (potentially many) structures in the same colour of $\mathcal{M}$; the exception is probability distributions in $\color{gray}\mathcal{P}$ which are generated by (potentially many) structures in $\color{red}\mathcal{M}$ and in $\color{yellow}\mathcal{M}$. $P$ is a probability distribution in $\mathcal{P}$ with probability distributions defined similarly for each colour in $\mathcal{P}$.}
\end{subfigure}
\begin{subfigure}{0.8\textwidth}
  \centering
  \includegraphics[width=\textwidth]{U:/GitHub/Covariatesets/Diagrams/Partial.pdf}
  \caption{That probability distributions in $\color{gray}\mathcal{P}$ are generated by structures in $\color{red}\mathcal{M}$ and in $\color{yellow}\mathcal{M}$ creates a complication; the value of $\chi$ is not constant across observationally equivalent structures that $\mathcal{M}$ admits and that generate a probability distribution in $\color{gray}\mathcal{P}$. So $\mathcal{M}$ does not uniformly identify $\chi$. Consideration of the identifying correspondence $G$ determines that this corresponds to there being structures in $\mathcal{M}$ for which $G$ does not deliver the value of $\chi$ when applied to the probability distributions that these structures generate. Nonetheless, if $\mathcal{M}$ restricts the set of values of $\chi$ for any probability distribution in $\mathcal{P}$ then $\mathcal{M}$ does have some non-trivial identifying power for $\chi$. Then $\mathcal{M}$ is said to uniformly partially identify $\chi$ if $\mathcal{M}$ and $\mathcal{P}$ can each be partitioned into countably many disjoint subsets and that a probability distribution in a partition of $\mathcal{P}$ is not generated by a structure in at least one partition of $\mathcal{M}$, holding for any such partition of $\mathcal{P}$. In the context of Figure~\ref{fig:partials} $\color{red}\mathcal{M}$ identifies $\chi$ up to $\lbrace a,c\rbrace$, $\color{blue}\mathcal{M}$ identifies $\chi$ uniquely to $b$, and $\color{yellow}\mathcal{M}$ identifies $\chi$ up to $\lbrace a,c\rbrace$. Each partition of $\mathcal{P}$ includes probability distributions that are generated by structures in at least one partition of $\mathcal{M}$. Equivalently, if $G$ is permitted to be a multivalued functional (or one-to-many) then $\mathcal{M}$ uniformly partially identifies $\chi$ if $G$ exists and if $G(P)$ contains the set of values of $\chi$ that are delivered by structures that generate $P$, holding for all such $P$. A caveat must be applied here; $G$ cannot be trivial in the sense that it is constant across all such $P$. Clearly this definition of $G$ does not exclude the possibility that there is multiplicity of identifying correspondences that satisfy this property. Sharpness is a desirable property in such circumstances; a functional $G$ that can be shown to deliver smaller sets according to some well-defined distance measure across all possible $P$ (and that satisfies the properties above) should be preferred to any alternative identifying correspondence.}
  \label{fig:partial1}
  \end{subfigure}
  \caption{Partial identification of a structural characteristic.}
  \label{fig:partials}
\end{figure}
%==================================================%
%% Bibliography.
\bibliographystyle{chicago}
\bibliography{\details Bibliography}
\end{document}

%I explore the effect of incorporating information on the identified set of values for the average causal effect of an endogenous variable on an outcome variable (the parameter of interest) that is delivered by a non-parametric model that embeds an exclusion restriction and an independence restriction that together characterise an instrumental variable. Endogenous variation enters the model since agents are permitted to non-randomly select a scalar observable characteristic. I consider the effect of combining many instrumental variables into a composite instrumental variable with many points of support on the identified set of values for counterfactual outcome distributions. Further, I consider the effect of enriching individual behaviour by allowing relevant exogenous variables to affect individual choice; specifically, I allow these variables to enter the structural equations that determine the endogenous variable and that determine the outcome variable. I establish the conditions under which the parameter of interest in the enriched model is equivalent to the parameter of interest in a model that does not explicitly account for the contribution of additional relevant exogenous variables. 
%
%The model that I consider partially identifies the parameter of interest. That is, the restrictions on the set of admissible structures (or data generating processes) that are implied by the model are insufficient to exclude observationally equivalent structures that deliver different values of the parameter of interest. The restrictions that are implied are nonetheless sufficient to restrict the set of values of the parameter of interest up to a non-trivial set. This concept is described graphically in Figure~\ref{fig:partials}. 
%\section{A}
%The model that I consider is partially identifying. That is, the restrictions on the set of admissible structures (or data generating processes) that are implied by the model are insufficient to exclude observationally equivalent structures. Furthermore, the model is unable to identify the structural characteristic of interest 
%
%\section{B}
%I explore the effect of incorporating information on the identified set of values for the average causal effect of an endogenous variable on an outcome variable (the parameter of interest) that is delivered by a non-parametric model. The model is weakly restrictive in that it does not embed a structural equation 
%
%
% that embeds an exclusion restriction and an independence restriction that together characterise an instrumental variable. The model is weakly restrictive in that it 
% 
%\section{C}
%Points to make:
%\begin{itemize}
%\item Non-parametric IV model with endogenous variable
%\item Non-random selection
%\item Average causal effect of endogenous variable on outcome variable is parameter of interest
%\item Model partially identifies the parameter of interest
%\item Model is weakly restrictive in that it does not specify a distribution for unobserved heterogeneity (imposes a normalisation and allows all distributional assumption to enter the non-parametric threshold function) and does not specify a selection equation
%\item Credible identification
%\item Want to incorporate additional information for the purpose of efficiency and to potentially provide tighter bounds on parameter of interest; may also be that conditional parameter of interest is useful
%\end{itemize}
%Related literature:
%\begin{itemize}
%\item \cite{cr13} The starting point for analysis
%\item \cite{crs13} Characterise level sets and identification
%\item \cite{bp97} Binary treatment characterise the identified set under non-compliance; weaker restrictions than presented here
%\item \cite{kI09} Continuous outcome variable; nesting of identified set under different model assumptions
%\item \cite{bLpO03} Average structural function when continuous endogenous regressor
%\item \cite{bEmOImOF12} Random set theory
%\end{itemize}