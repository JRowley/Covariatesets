%% Page settings.
\documentclass[10pt,a4paper,twoside]{article}
\usepackage[top=1.2in, left=1.2in, bottom=1.2in, right=1.2in]{geometry}
%==================================================%
%% Encoding packages.
\usepackage[UKenglish]{babel}
\usepackage[nodayofweek]{datetime}
\usepackage[T1]{fontenc}
\usepackage[utf8]{inputenc}
\usepackage{amsmath}
\usepackage{amsthm}
\DeclareMathAlphabet\mathbb{U}{msb}{m}{n}
%\usepackage{amsfonts}
\usepackage{amssymb}
\usepackage{calc}
\usepackage{natbib}
\usepackage{color}
\usepackage{subcaption}
%==================================================%
%% Document details.
\usepackage{titling}
\title{Information and admissible sets}
\author{Jeff Rowley}
\newcommand{\thedate}{\today}
% Enter document details here.
\newcommand{\details}{C:/Dropbox/TeXTemplates/}
% Enter the file path here for the UCL logo and bibliography.
% Change this file path for different computer systems.
\newcommand{\homePC}{C:/Users/Jeffro/Documents/}
\newcommand{\workPC}{U:/}
%==================================================%
%% Date macro.
\newcommand{\theseason}[1]{
\ifcase \month 
\or Winter\or Winter\or Spring\or Spring\or Spring\or Summer\or Summer\or Summer\or Autumn\or Autumn\or Autumn\or Winter\fi}
% Displays the season.
% Obsolete in this template.
%==================================================%
%% Useful packages.
\usepackage{enumerate}		% Lists.
\usepackage{bbm}			% Indicator functions.
\usepackage{lipsum}			% Random text generator.
\usepackage{MnSymbol}		% Arrows.
\usepackage{graphicx}		% Graphics.
\usepackage{mathtools}		% Math*lap command.
%==================================================%
%% Theorem environments.
\newcounter{countthm}[section]
\newcounter{countlem}[section]
\newcounter{countex}[section]
% Creates new counters which reset at each new section. 
\renewcommand{\thecountthm}{\thesection.\arabic{countthm}}
\renewcommand{\thecountlem}{\thesection.\arabic{countlem}}
\renewcommand{\thecountex}{\thesection.\arabic{countex}}
% Redefines counters to include the section number.
\newtheorem{thm}[countthm]{Theorem}
% Creates a theorem environment - type \thm to begin.
\newtheorem{lem}[countlem]{Lemma}
% Creates a lemma environment - type \lem to begin.
\newtheorem{ex}[countex]{Example}
% Creates an example environment - type \ex to begin.
\newtheorem*{Acknowledgements}{Acknowledgements}
% Creates a thanks environment.
\newcommand{\newthm}[1]{\newtheorem*{#1}{#1}}
% A macro that makes defining new theorem environments quick.
% Type \newthm{<Theorem name here>} to begin.
%==================================================%
%% Math operators.
\DeclareMathOperator*{\plim}{plim}
% Writes plim in math environment - type \plim to enter.
\DeclareMathOperator*{\argmax}{argmax}
% Writes argmax in math environment - type \argmax to enter.
\DeclareMathOperator*{\argmin}{argmin}
% Writes argmin in math environment - type \argmin to enter.
\DeclareMathOperator*{\argsup}{argsup}
% Writes argsup in math environment - type \argsup to enter.
\DeclareMathOperator*{\arginf}{arginf}
% Writes arginf in math environment - type \arginf to enter.
\newcommand\independent{\protect\mathpalette{\protect\independenT}{\perp}} 
\def\independenT#1#2{\mathrel{\rlap{$#1#2$}\mkern2mu{#1#2}}} 
% Pastes an independence symbol - type \independent to enter.
\DeclareMathOperator*{\generates}{:.}
% Write generate symbol (identification).
\DeclareMathOperator*{\generated}{.:}
% Write generated symbol (identification).
%==================================================%
%% Abbreviations.
\newcommand{\US}{United States}
% Use \US macro when use is as a noun.
% Use U.S. when use is as an adjective.
%==================================================%
%% Equation numbering.
\numberwithin{equation}{section}
% Numbers equations up to the subsection. To change level,
% replace subsection with section.
%==================================================%
%% Counters.
\newcounter{saveenumi}
\newcounter{saveenumi1}
\setcounter{section}{0}
%==================================================%
\newcommand{\ESRC}{I gratefully acknowledge financial support from the Economic and Social Research Council (ESRC).}
\newcommand{\RLaTeX}{I acknowledge the \emph{R} and \emph{\LaTeX} communities, and the wealth of knowledge that they have made freely available to all.}
%==================================================%
%% The title page.
\makeatletter				
% Changes @ to catcode 11.
\renewcommand{\@maketitle}{
\null
\graphicspath{ {\details} }
\flushleft{\includegraphics[width=40mm]{UCL_Logo_Orange}}
\hspace{5mm}
\normalsize Department of Economics, University College London\\
\vskip\bigskipamount
\leaders\vrule width \textwidth\vskip0.4pt 
\vskip\bigskipamount 
\nointerlineskip
% This completes the UCL banner.
\begin{center}
\begin{minipage}{100mm}
\begin{center}
\vspace{20mm}
\LARGE
\textbf{
\@title}
\par
\vspace{10mm}
\normalsize
\@author
\par
\vspace{5mm}
\normalsize
\thedate
\end{center}
\end{minipage}
\end{center}
}
\makeatother				
% Reverts @ to catcode 12.
%==================================================%
%% Packages to load at end of preamble. 
% Note conflict between Tkz-euclide package set and 
% game theory package set. Load one or the other.
\usepackage{hyperref}
%\usepackage[numbered]{mcode}

%% Tkz-euclide package set.
%\usepackage{tkz-euclide}
%\usepackage{pgfplots}
%\usepgfplotslibrary{external} 
%\tikzexternalize[prefix=tikz/]

%% Game theory package set.
\usepackage{pstricks}	
\usepackage{egameps}		 
\usepackage{pst-3d}			
\usepackage{sgame}	
\renewcommand{\gamestretch}{1.5}
%==================================================%
%% Further notes regarding egameps package.
% The egameps package is incompatible with this template
% due to UCL logo. Solution is to independently run 
% egameps through on a latex blank document, then insert 
% pdf into this document. Recall that to run egameps:
% "latex" --> "DVi->PS" --> "PS->PDF"
% then use "includegraphics[]" with trim option. 
%==================================================%
%% Headers and footers.
\usepackage{fancyhdr}
\pagestyle{fancy}
\renewcommand{\sectionmark}[1]{\markright{\thesection.\ #1}}
% This redefines the \rightmark command so that the section number does not appear.
% NOTE: To remove the section number, delete <<\thesection.\>>
\lhead[\thepage]{\rightmark}
\rhead[\rightmark]{\thepage}
\chead[]{}
\cfoot[]{}
%\lfoot[\thetitle]{}
%\rfoot[]{\theauthor}
\renewcommand*\thesection{\arabic{section}}
\usepackage{epigraph}
%==================================================%
%% Start of document.
\begin{document}
\maketitle
\vspace{10mm}
\begin{abstract}
\noindent <<Abstract here>>
\begin{Acknowledgements}
{\RLaTeX} I thank Andrew Chesher and Toru Kitagawa for their supervision and support. I further thank Adam Rosen for helpful discussion. {\ESRC}
\end{Acknowledgements}
\end{abstract}
\vspace{5mm}
%==================================================%
%% Document.
%==================================================%
I explore the effect of incorporating information for a non-parametric binary choice model. The model permits endogenous variation in a scalar random variable that is due to non-random selection, and it is the average causal effect of this endogenous variable on the outcome variable that is of interest. The model embeds an exclusion restriction and an independence restriction that together define an instrumental variable but is silent as to the relationship between the endogenous variable and the instrumental variable. I restrict the relationship between the outcome variable and the endogenous variable up to a non-parametric threshold crossing function. The model is credible \citep{book.manski} in that it embeds restrictions that impose weaker constraints on assumed behaviour, but does not identify the average causal effect of the endogenous variable on the outcome variable.\footnote{Assumptions that cannot be tested using data. The model does embed some non-trivial non-verifiable restrictions that might be relaxed.} Rather, the model partially identifies the average causal effect of the endogenous variable on the outcome variable. 

I define information to be those additional characteristics of economic agents that are observable with the caveat that these characteristics are exogenous and relevant are to the latent structure. It is convenient to think of such characteristics as being predetermined and immutable; characteristics that result from choices that are made jointly with the outcome variable are excluded by the definition. Accordingly, exogenous variables and instrumental variables are each regarded as information, and I distinguish between these classes of information. I study how the admissible set of values for the average causal effect of the endogenous variable on the outcome variable changes as each class of information is incorporated into the model separately. 

It is useful to distinguish between classes of information since each class enters the latent structure in a different way. Exogenous variables are permitted to enter the structural equation for the outcome variable and to determine the endogenous variable. As such, exogenous variables can be seen to enrich both individual response and individual selection, respectively. An important consequence is that the causal effect of the endogenous variable on the outcome variable depends upon the value of the exogenous variables when individual response is enriched. In contrast, instrumental variables are excluded from the structural equation for the outcome variable by definition and so only enrich individual selection. Given this, the effect of incorporating information is different depending upon the class of information that is being incorporated into the model.  

Incorporating information of either class is generally sensible for a number of reasons. Firstly, incorporating information is known to be efficient; variation that is attributable to an observable variable is instead attributable to unobservable heterogeneity when that variable is omitted. Secondly, the effect of incorporating information for partially identifying models is not well-documented; one hypothesis is that incorporating information narrows bounds on admissible sets. Such an effect is not documented in identifying models precisely because such models deliver a point estimate (a set of length zero), but point estimates may shift as information is incorporated. A contribution that I make is in showing that \color{red} incorporating information leads to narrower bounds on the admissible sets \color{black} that are delivered by the model. A further reason to particularly favour incorporating exogenous variables is that the average causal effect of the endogenous variable on the outcome variable in identifiable sub-populations can be recovered. I name this structural characteristic the conditional average causal effect of the endogenous variable on the outcome variable, and index it by the conditioning value.\footnote{The conditioning value is specifically the value of the exogenous variables. \cite{hEvY05} defines a parameter $ATE(x)$ that is equivalent to the conditional average causal effect of the endogenous variable on the outcome variable at the conditioning value $x$. \cite{kHt10} and \cite{13.misc.abrevaya} instead refer to this parameter as the conditional average treatment effect and abbreviate this to $CATE(x)$.} Understanding the effect of an intervention in sub-populations can be interesting if the intervention can be targeted or if the intervention is to be applied elsewhere in a population that differs according to its observable characteristics. 

A relevant question is how to relate conditional causal effects to (unconditional) causal effects. More precisely, how does the average causal effect of the endogenous variable on the outcome variable relate to its conditional counterparts? I show that the average causal effect of the endogenous variable on the outcome variable can be expressed as a Minkowski summation of its conditional counterparts when the non-parametric binary choice model is augmented. I derive sharp bounds on the conditional average causal effect by applying random set theory. I employ the capacity (or containment) functional as in \cite{crs13} as a matter of choice, rather than the Aumann expectation as in \cite{bEmOImOF12}. As I show that the average causal effect of the endogenous variable on the outcome variable can be expressed as a Minkowski summation of its conditional counterparts, I derive sharp bounds on the average causal effect of the endogenous variable on the outcome variable.\footnote{\cite{book.molchanov} is a useful companion in the study of random sets.} I establish the conditions under which bounds on conditional causal effects can be informative about bounds on (unconditional) causal effects. That is, I establish the conditions under which bounds on conditional causal effects can be used to narrow bounds on (unconditional) causal effects, exploiting the mapping from one to the other.      

I demonstrate application of the non-parametric binary choice model, elucidating the practical difficulties that arise when estimating set identifying models (focusing on those issues that arise from incorporating information). As in \cite{cr13}, I estimate the average causal effect of childbirth on a mother's labour force participation using US census data. I extend \cite{cr13} in a number of ways. Firstly, I report statistical uncertainty in the estimate of the average causal effect of childbirth on a mother's labour force participation using a method that is outlined in \cite{cHlr13}. Secondly, I enrich the support of the instrumental variable and explore the effect that this has on the admissible set of values for the average causal effect of childbirth on a mother's labour force participation, and on its accompanying confidence region. Thirdly, I enrich individual response by permitting the structural equation for labour force participation to depend upon the age and other such predetermined and immutable characteristics of mothers. I discuss the complication of calculating statistical uncertainty when exogenous variables are permitted to enter the structural equation for labour force participation. With respect to the second and third extensions, it is necessary that I augment the model by embedding additional restrictions. In fact, \cite{cr13} describe the augmented non-parametric binary choice model that I assume but simplify this model for application (by excluding exogenous variables from the structural equation for the outcome variable). I discuss how the augmented model relates to the simplified model in each case and the credibility of the additional restrictions that are embedded in the augmented model.
%==================================================%
\section*{Related research}
Other notable non-parametric binary choice models are described in \cite{bp97} and \cite{sHvY11}, and general non-parametric models of choice are described in \cite{c05}, \cite{kI09} and \cite{c10}.

\cite{bp97} assumes a triangular model (the model embeds a structural equation for the outcome variable and a structural equation for the endogenous variable; see \cite{sTw60} for a detailed discussion of triangular models) that relaxes separability of unobservable heterogeneity in the structural equation for the outcome variable. The cost is that the model is no longer silent as to the relationship between the endogenous variable and the instrumental variable. The model does not permit exogenous variables to enter the structural equation for the outcome variable. I discuss the credibility of separability of unobservable heterogeneity in the main text. \cite{sHvY11} assumes a triangular model but maintains separability of unobservable heterogeneity in the structural equation for the outcome variable. The model permits exogenous variables to enter the structural equation for the outcome variable.

\cite{c05} and \cite{kI09} describe non-parametric models that permit continuous variation in the outcome variable. \cite{c05} assumes a triangular model that relaxes separability of unobservable heterogeneity in the structural equation for the outcome variable. The model permits exogenous variables to enter the structural equation for the outcome variable, although local invariance of the structural equation for the outcome variable to variation in the exogenous variables is embedded. The model is uninformative when there is binary variation in the endogenous variable but is informative when there is discrete variation. \cite{kI09} extends \cite{bp97} to permit discrete and continuous variation in the outcome variable, and studies commonly invoked restrictions on covariation of the instrumental variable and unobservable heterogeneity.

\cite{c10} describes an ordered choice model that permits discrete variation in the outcome variable. \cite{c10} assumes a single equation model that relaxes separability of unobservable heterogeneity in the structural equation for the outcome variable, although monotonicity of the structural equation for the outcome variable in unobservable heterogeneity is embedded. The model permits binary or discrete variation in the endogenous variable.
%==================================================%
\section*{Notation}
I study a probability space $(\Omega,\Sigma,\mathbb{P})$. I define random variables on this probability space. I write random variables as upper case Latin letters, and I write realisations (or specific values) of random variables as lower case Latin letters. I write the support of $A$ as $\mathcal{R}_A$. I write the counterfactual value of $A$ when $B$ has a causal interpretation and is externally fixed as $A(b)$. I write the average causal effect of $B$ on $A$ as $ACE(B\rightarrow A)$, and the conditional average causal effect of $B$ on $A$ given $C$ as $ACE(B\rightarrow A|c)$.

I refer to $Y$ as the outcome variable, to $D$ as the endogenous variable, to $X$ as the endogenous variable, to $Z$ as the instrumental variable, and to $U$ as unobservable heterogeneity. Despite the use of \emph{the}, I permit $(X,Z)$ to be vectors. I write the structural equation for the outcome variable as $h$, and the structural equation for the endogenous variable as $g$.  

I write the expectation operator as $\mathbb{E}$, and the indicator function as $\mathbbm{1}$. I write $A$ is independent of $B$ as $A\independent B$. To distinguish between population and sample quantities, I subscript sample quantities by $n$. 

I introduce further terminology and notation in Figure~\ref{fig:models} through Figure~\ref{fig:partials}. This specifically relates to models and structures, and is consistent with the approach that is formally laid out in \cite{h50} and in \cite{krE50}. 
%==================================================%
\section*{Application}
I estimate the average causal effect of childbirth on a mother's employment using {\US} census data. The data are obtainable from \cite{Angristdatabank}, and are described in \cite{ae98}. To summarise, the dataset consists of 254,654 households that were recorded as part of the 1980 {\US} census. The dataset specifically contains observations of married households with at least two children under the age of 18 years and where the mother is aged between 21 years and 35 years. For clarity, I translate each variable in the data into the mathematical notation that I employ.
\begin{align*}
Y&\equiv\mathbbm{1}[\text{Mother is employed in 1979}]\\
D&\equiv\mathbbm{1}[\text{Three or more children in the household}]
\end{align*}
As $(X,Z)$ are continually redefined in the main text, I do not define these variables as I do $(Y,D)$. Instead, I note those variables that at some point or other form part of the definition of $(X,Z)$. $X$ is a function of mother's race or ethnicity (shortened to race).\footnote{\cite{ae98} also treats mother's age, and mother's age at the time of her first birth (shortened to birthing age) as exogenous variables.} $Z$ is a function of whether the oldest two children in the household share the same gender (shortened to child gender), and whether the second pregnancy was a multiple pregnancy.

I refer to the application throughout the main text so as to illustrate how technical conditions on variables and on the relationship between variables restrict the behaviour of economic agents, in this case mothers. For brevity, I simply refer to race when discussing $X$ in the context of the application, and child gender when discussing $Z$ in the context of the application.
%==================================================%
\section{A non-parametric model of binary choice}
\newthm{Axiom}
\begin{Axiom}
Economic agents are utility maximising, selecting between alternatives in a choice set according to the utility that they attach to that choice. Utility is perfectly observable by economic agents and is determined by a well-defined utility function for each choice. Each agent is permitted to value each choice differently. 
\end{Axiom}
\noindent I introduce the non-parametric binary choice model that is described in \cite{cr13}. The model constitutes the set of structures that are consistent with Restriction \small{M}\normalsize{1} through Restriction \small{M}\normalsize{5}. 
\begin{enumerate}[\bf \small{M}\normalsize{1}.] 
\item $\mathrlap{\text{\bf Discrete support.}}$\phantom{\textbf{Joint independence.}} $(Y,D,X,Z)$ are observable and have discrete supports (with at least two points of support). Further, $(Y,D)$ have binary supports and are normalised such that\begin{enumerate}[(a)]
\item $\mathcal{R}_Y=\lbrace 0,1\rbrace$ and
\item $\mathcal{R}_D=\lbrace 0,1\rbrace$,
\end{enumerate}
respectively.
\end{enumerate}
Restriction \small{M}\normalsize{1} is a verifiable restriction. $(Y,D,X,Z)$ are observable and so it is trivial to verify that each variable satisfies its support restriction. $(Y,D)$ are normalised to be consistent with the application, but any other supports $\lbrace y_0,y_1\rbrace$ and $\lbrace d_0,d_1\rbrace$ can be generated by an affine transformation of $h$ and of $g$. 
\begin{enumerate}[\bf \small{M}\normalsize{2}.] 
\setcounter{enumi}{2}
\item $\mathrlap{\text{\bf Scalar $\boldsymbol{U}$.}}$\phantom{\textbf{Joint independence.}} $U$ is an unobservable scalar such that $\mathcal{R}_U$ is an open subset of $\mathbb{R}$ with strictly positive Lebesgue measure.
\end{enumerate}
Restriction \small{M}\normalsize{2} is a non-verifiable restriction. The restriction is not overly restrictive in that it is equivalent to those determinants of utility that are unobservable being defined on a set of cardinality no greater than $2^{\aleph_0}$. As most economic variables are defined on $\mathbb{R}$ or on a subset of $\mathbb{R}$ it is uncontroversial to assume that this restriction is satisfied. In the context of the application, the restriction implies that variables such as job application ability and taste for leisure are quantifiable and can be defined on $\mathbb{R}$. 
\begin{enumerate}[\bf \small{M}\normalsize{3}.] 
\setcounter{enumi}{2}
\item $\mathrlap{\text{\bf Joint independence.}}$\phantom{\textbf{Joint independence.}} $U\independent (X,Z)$.
\end{enumerate}
Restriction \small{M}\normalsize{3} is a non-verifiable restriction. The restriction nests the restrictions $U\independent X$ and $U\independent Z$ (marginal independence), and is the strongest possible restriction that might be imposed upon the joint distribution of $(X,Z,U)$. The restriction excludes some correlation structures of $(X,Z)$ that are permitted by marginal independence. In the context of the application, the restriction implies that variables such as job application ability and taste for leisure are independent of combinations of race and child gender, and are independent of race and of child gender. 
\begin{enumerate}[\bf \small{M}\normalsize{4}.] 
\setcounter{enumi}{3}
\item $\mathrlap{\text{\bf Exclusion.}}$\phantom{\textbf{Joint independence.}} $Y\equiv h(D,X,U)$.
\end{enumerate}  
Restriction \small{M}\normalsize{4} is a non-verifiable restriction. The restriction excludes $Z$ from $h$ and so excludes $Z$ from having a causal effect on $Y$. The restriction is equivalent to an order condition. In the context of the application, the restriction implies that child gender does not have a causal effect on female employment.  
\begin{enumerate}[\bf \small{M}\normalsize{5}.] 
\setcounter{enumi}{4}
\item $\mathrlap{\text{\bf Monotonicity.}}$\phantom{\textbf{Joint independence.}} $h$ is a non-parametric threshold crossing function that is normalised to be increasing in $U$. $U$ is normalised to be distributed uniformly on $[0,1]$.
\end{enumerate} 
Restriction \small{M}\normalsize{5} is a non-verifiable restriction. The restriction implies that individual response is monotonic. In the context of the application, the restriction implies that the causal effect of childbirth on a mother's employment is positive for all mothers or is negative for all mothers. The restriction permits the threshold to be a non-parametric function of $(D,X)$ and implies that the distribution of $U$ can be relatively unrestricted beyond Restriction \small{M}\normalsize{2}.  

Restriction \small{M}\normalsize{1} through Restriction \small{M}\normalsize{5} can be written more compactly as Restriction \small{M}\normalsize{1}' through Restriction \small{M}\normalsize{5}'. Restriction \small{M}\normalsize{1}' particularly makes clear the relationship between $Y$ and $(D,X,Z,U)$. 
\begin{enumerate}[\bf \small{M}\normalsize 1'.]
\item $Y=\mathbbm{1}[p(D,X)>U]$.
\item $U\vert(X,Z)\sim\text{unif}(0,1)$.
\item $\mathcal{R}_D=\lbrace 0,1\rbrace$.
\item $\mathcal{R}_X=\lbrace x_1,...,x_K\rbrace$ and $K<\infty$.
\item $\mathcal{R}_Z=\lbrace z_1,...,z_L\rbrace$ and $L<\infty$.
\end{enumerate}
%==================================================%
\section{Credibility in economic modelling}
I discuss the advantage of a single equation model over more restrictive models, and the advantage of a single equation model over reduced form analysis. I discuss the credibility of Restriction \small{M}\normalsize{3} through Restriction \small{M}\normalsize{5}, theoretically and in the context of the application. 

Credibility is a statement of the validity and the plausibility of the restrictions that a model embeds, and is a desirable property. The need to discuss both validity and plausibility arises because restrictions can be either verifiable or non-verifiable. The distinction between verifiable and non-verifiable restrictions is that verifiable restrictions are testable using data while non-verifiable restrictions cannot be tested even if data is collected for the population. As verifiable restrictions can be rejected or not rejected on the basis of observed behaviour, it makes sense to talk about such restrictions as being valid or invalid. In contrast, the validity of non-verifiable restrictions is indeterminable. Whether to accept a set of non-verifiable restrictions as an accurate representation of how economic agents behave is subjective and depends upon how plausible the restrictions seem. Restrictions that are founded in economic theory, or that impose weaker constraints on assumed behaviour are more plausible (a view that is consistent with Occam's razor, a widely accepted principle of parsimony).      

I regard a model as incredible if the verifiable restrictions that it embeds are invalid. I regard a model as more credible relative to another if the verifiable restrictions that it embeds are valid, and if the non-verifiable restrictions that it embeds are more plausible. \cite{book.manski} adopts an equivalent stance, formalised as The Law of Decreasing Credibility. 

Models that embed restrictions that impose weaker constraints on assumed behaviour are typically not uniformly identifying. Instead, such models are typically partially identifying. More commonly, 
\begin{enumerate}[(a)]
\item a more restrictive model is assumed that identifies a feature of interest;
\item or, identification of a different feature is sought and a model that embeds restrictions that impose weak constraints on assumed behaviour is assumed.\footnote{The local average treatment effect \citep{ai94} is an example of a feature that is identified by a model that embeds restrictions that impose weak constraints on assumed behaviour.} 
\end{enumerate}
I suggest that these responses to partial identification are motivated by two concerns. Namely, that characterising the admissible set of structures or the admissible set of values for a structural characteristic of interest can be complex and computationally difficult, and that partially identifying models do not produce unique conclusions. Although tractability is a legitimate concern, there is an inherent and widespread misunderstanding that models that do not produce unique conclusions are inferior regardless of the restrictions that they embed. Conclusions that are produced by more credible models should always be preferred, even if these conclusions display ambiguity. 

I caution against both responses. In the first response, a more restrictive model is assumed specifically for the purpose of achieving identification. \cite{krE50} remarks that a model should be constructed purely from prior knowledge of the studied behaviour, and to do otherwise violates scientific honesty. \cite{book.manski} remarks that this response displays incredible certitude. In the second response, the feature that is identified is often less interesting and has less value than the original feature of interest. For example, the local average treatment effect is informative of the effect of an intervention for those economic agents that are affected but this sub-population is not identifiable. The local average treatment effect is only useful for \emph{ex-post} evaluation. Nonetheless, it is promising that the second response should implicitly recognise the importance of credibility.

To emphasise the advantage of a single equation model, I introduce two models. The first is described in \cite{hE78} and is an example of a more restrictive model (the multivariate probit model). The second is described in \cite{ai94} and is an example of a model that identifies a different feature. 

\begin{enumerate}[\bf \small{H}\normalsize 1.]
\item $Y=\mathbbm{1}[X'\alpha+D\beta>U_Y]$.
\item $D=\mathbbm{1}[X'\delta+Z\gamma>U_D]$.
\item $(U_Y,U_D)|(X,Z)\sim\mathcal{N}(\mu,\Sigma)$.
\end{enumerate}
A single equation model is more credible than the multivariate probit model. Firstly, the multivariate probit model restricts the relationship between variables to have a parametric and linear index representation (Restriction \small{H}\normalsize{1} and Restriction \small{H}\normalsize{2}). This restriction is not entirely implausible as economic agents might well perform reasonably simple calculations, but is not trivial and excludes some forms of individual response and individual selection that are permitted by more flexible non-parametric index representations. Secondly, the multivariate probit model restricts the distribution of $U$ to have a parametric distribution (Restriction \small{H}\normalsize{3}). A single equation model is not free of distributional restrictions in that it restricts the relationship between $U$ and $(X,Z)$, but the multivariate probit model goes beyond this by restricting $U$ to belong to the class of parametric distributions. The restriction implies that the relationship between variables is fully specified. There is no reason to suppose that $U$ should be parametric, nor that there is full knowledge of the process that generates endogeneity. Non-verifiable restrictions that impose a distribution on $U$ are particularly implausible, and models that embed such restrictions are incredible.
\begin{enumerate}[\bf \small{A}\normalsize 1.]
\item $Y=\mathbbm{1}[X'\alpha+D\beta>U_Y]$.
\item $D=\mathbbm{1}[X'\delta+Z\gamma>U_D]$.
\item $(U_Y,U_D)|(X,Z)\sim\mathcal{N}(\mu,\Sigma)$.
\end{enumerate}


%Further, the fact that partially identifying models do not produce unique conclusions regarding how economic agents behave 
%
%As more credible models impose weaker constraints on assumed behaviour they 
%
%
%Credibility is a statement of the validity and plausibility of the restrictions that a model embeds, and is a desirable property. Specifically, the validity of verifiable restrictions and the plausibility of non-verifiable restrictions. If a model embeds verifiable restrictions that are incompatible with data then it is invalid and incredible, else it is valid with its credibility dependent upon the plausibility of non-verifiable restrictions. I regard non-verifiable restrictions that are consistent with economic theory or that are less restrictive as more plausible. This approach is consistent with the law of decreasing credibility \citep{book.manski} and with Occam's razor.
%
%A discouraging quality of more credible models is that they often do not uniformly identify structures or structural characteristics of interest. Typically, such models are partially identifying due to the weak restrictions that they embed. Characterising the admissible set of structures or the admissible set of values for a structural characteristic of interest can be complex and computationally difficult. Further, the fact that more credible models do not produce unique conclusions regarding the nature of economic behaviours is dissuasive. Although tractability is a legitimate concern, assuming a more restrictive model in order to produce a unique conclusion is not. \cite{book.manski} states that estimation and inference that is produced by more restrictive models than is necessary displays incredible certitude. I suggest that it is an inherent and widespread misunderstanding that conclusions that display ambiguity are inferior to conclusions that do not; conclusions that are produced by a more credible model are more informative than conclusions that are produced by a less credible model, regardless of whether they display ambiguity or not. 
%
%A single equation model is generally preferable to a more restrictive model that embeds a structural equation for the endogenous variable, and more stringent restrictions on the structural equation for the outcome variable. 


%The restrictions that are embedded in an economic model can be categorised as verifiable and non-verifiable. Verifiable restrictions are those that can be verified using data, while non-verifiable restrictions are those restrictions that can never be verified even if data is collected for the population. Verifiable restrictions are an obvious basis on which to select between competing models since if data is incompatible with a verifiable restriction then any model that embeds that restriction must be incompatible with the data generating process. Credibility then rests upon the non-verifiable restrictions that a model embeds. The more stringent are the non-verifiable restrictions that a model embeds then the less credible is the model. The exception is if a non-verifiable restriction is motivated by economic theory (monotonicity, say). Occam's razor promotes the ideal of parsimony.
%
%A subsequent question is then what are the implications of the credibility of economic models for the estimates that they deliver? Logic dictates that if a model is preferred to another then any estimates that are delivered by the first model should be preferred to the estimates that are deliver by the less preferred model. This is regardless of the information that is contained in the estimates. At the extreme, if a model is considered to be incredible then the estimates that the model delivers should be rejected.  
%
%Restriction \small{M}\normalsize{3} is a particularly stringent restriction. 
% 
%The credibility of this assumption is questionable in that job application ability may be correlated with race (if employer discrimination systematically favours particular races), and taste for leisure may be correlated with child gender (if parents value leisure more when their children are of mixed gender). I suggest that the credibility of the restriction is no more questionable than marginal independence in the context of the application since it seems unlikely that either unobservable variable is correlated with combinations of race and child gender, rather than with race or with child gender.
%==================================================%
\section{Incorporating information}
\subsection{Enriching the support of the instrumental variable}
\subsection{Enriching individual response}
%==================================================%
%\section{A non-parametric model of binary choice}
%I begin by introducing the non-parametric binary choice model that is studied in \cite{cr13}. I discuss the credibility of the restrictions that 
%are embedded in the model and the limitations of these restrictions. I then augment the model and incorporate information. Firstly, I enrich the support of the instrumental variable. Secondly, I enrich the structural function for the outcome variable by allowing individual response to depend upon exogenous variables as well as on the endogenous variable. In both cases, I discuss the credibility of the restrictions that are embedded in the augmented model and the limitations of these restrictions. I discuss the effect of incorporating both classes of information on the admissible set of values for the average causal effect of the endogenous variable on the outcome variable. 
%\subsection{A baseline model}
%\cite{cr13} 
%\begin{align*}
%Y&=\mathbbm{1}[p(D)>U]\\
%U&\independent Z
%\end{align*}
%==================================================%
%% Diagrams
%==================================================%
\begin{figure}[p]
\centering
\begin{subfigure}{0.8\textwidth}
  \centering
  \includegraphics[width=\textwidth]{U:/GitHub/Covariatesets/Diagrams/Identification.pdf}
  \caption{A model $\mathcal{M}$ is a set of structures that forms a proper subset of the class of all structures $\mathcal{S}$. Each structure in $\mathcal{M}$ generates a probability distribution in the class of all probability distributions (of observable variables) $\mathcal{P}$. Then the image $\mathcal{I}$ is the set of all probability distributions that are generated by structures in $\mathcal{M}$.}
  \label{fig:model}
  \end{subfigure}
\begin{subfigure}{0.8\textwidth}
  \centering
  \includegraphics[width=\textwidth]{U:/GitHub/Covariatesets/Diagrams/Observationalrestrictiveness.pdf}
  \caption{A structure $S$ is incompatible with data if it generates a probability distribution (of observable variables) $P$ that is distinct from a realised probability distribution $\mathbb{P}$. If all structures in $\mathcal{M}$ are incompatible with data then $\mathcal{M}$ is said to be observationally restrictive, and is falsified. This condition is equivalent to $\mathbb{P}\in\mathcal{P}\setminus\mathcal{I}$.}
  \label{fig:obs.restrict}
  \end{subfigure}
\caption{Structures, models, probability distributions (of observable variables), and falsifiability.}
\label{fig:models}
\end{figure}
%==================================================%
\begin{figure}[p]
\centering
\begin{subfigure}{0.8\textwidth}
  \centering
  \includegraphics[width=\textwidth]{U:/GitHub/Covariatesets/Diagrams/Pointidentification.pdf}
  \caption{A model $\mathcal{M}$ is said to identify a structure $S$ if the probability distribution (of observable variables) $P$ that is generated by $S$ is distinct from those generated by other structures in $\mathcal{M}$. The structures $S_a$, $S_b$ and $S_c$ are said to be observationally equivalent as they all generate $P$ but $S_b$ and $S_c$ are not admitted by $\mathcal{M}$. As $S_a$ is the only structure that is admitted by $\mathcal{M}$ and that generates $P$, $S_a$ is identified by $\mathcal{M}$. For completeness, $\mathcal{M}$ is said to be uniformly identifying if it identifies each structure that it admits.}
  \label{fig:identify}
  \end{subfigure}
  \begin{subfigure}{0.8\textwidth}
  \centering
  \includegraphics[width=\textwidth]{U:/GitHub/Covariatesets/Diagrams/Setidentification.pdf}
  \caption{As $S_a$ and $S_b$ are observationally equivalent and are both admitted by $\mathcal{M}$ then $\mathcal{M}$ does not identify either $S_a$ or $S_b$. Nonetheless, as $\mathcal{M}$ restricts the set of observationally equivalent structures that generate $P$ to $S_a$ and $S_b$ then $\mathcal{M}$ partially identifies $S_a$ (and $S_b$ to within $\lbrace S_a,S_b\rbrace$).}
  \label{fig:partial}
  \end{subfigure}
\caption{Identification and non-identification of a structure, and partial identification of a structure.}
\label{fig:identification}
\end{figure}
%==================================================%
\begin{figure}[p]
\centering
\begin{subfigure}{0.8\textwidth}
  \centering
  \includegraphics[width=\textwidth]{U:/GitHub/Covariatesets/Diagrams/Characteristic.pdf}
  \caption{A structural characteristic $\chi$ is a function of a structure $S$. A model $\mathcal{M}$ can be partitioned such that structures in a partition deliver the same value for $\chi$. Structures in the red partition $\color{red}\mathcal{M}$ deliver the value $a$ for $\chi$, and structures in the red partition $\color{blue}\mathcal{M}$ deliver the value $b$ for $\chi$. If $\chi$ is constant across all observationally equivalent structures that $\mathcal{M}$ admits then $\mathcal{M}$ is said to identify $\chi$. As $\chi(S_a)$ is equal to $\chi(S_b)$ (is equal to $a$) $\mathcal{M}$ identifies $\chi$.}
  \label{fig:characteristic}
  \end{subfigure}
  \begin{subfigure}{0.8\textwidth}
  \centering
  \includegraphics[width=\textwidth]{U:/GitHub/Covariatesets/Diagrams/Uniform.pdf}
  \caption{If $\mathcal{M}$ identifies $\chi$ for all structures in $\mathcal{M}$ then $\mathcal{M}$ is said to uniformly identify $\chi$. The class of all probability distributions (of observable variables) is partitioned into the blue partition $\color{red}\mathcal{P}$ and into the red partition $\color{blue}\mathcal{P}$. Probability distributions in $\color{red}\mathcal{P}$ are generated by (potentially many) structures in $\color{red}\mathcal{M}$, and probability distributions in $\color{blue}\mathcal{P}$ are generated by (potentially many) structures in $\color{blue}\mathcal{M}$. It is important that the number of partitions in $\mathcal{M}$ and in $\mathcal{P}$ are equal, although that number can be countably infinite. In the context of Figure~\ref{fig:uniform} $\mathcal{M}$ uniformly identifies $\chi$ since observationally equivalent structures that $\mathcal{M}$ admits are in the same colour of $\mathcal{M}$. More conveniently, whether $\mathcal{M}$ uniformly identifies $\chi$ can be determined by the existence of an identifying correspondence $G$, a functional. $\color{red}P$ is a probability distribution in $\color{red}\mathcal{P}$, and $\color{blue}P$ is a probability distribution in $\color{blue}\mathcal{P}$. Then $\mathcal{M}$ uniformly identifies $\chi$ if the value of $G(\color{red}P\color{black})$ is $a$ and if the value of $G(\color{blue}P\color{black})$ is $b$, holding for any such $\color{red}P$ and $\color{blue}P$. Notice that if $\mathcal{M}$ uniformly identifies all $\chi$ then $\mathcal{M}$ also uniformly identifies structures.}
  \label{fig:uniform}
  \end{subfigure}
  \caption{The identification of structural characteristics, and identifying correspondences.}
  \label{fig:characteristics}
\end{figure}
%==================================================%
\begin{figure}[p]
\centering
\begin{subfigure}{0.8\textwidth}
\centering
\caption{A structural characteristic $\chi$ is a function of a structure $S$. A model $\mathcal{M}$ can be partitioned such that structures in a partition deliver the same value for $\chi$. Structures in the red partition $\color{red}\mathcal{M}$ deliver the value $a$ for $\chi$, structures in the blue partition $\color{blue}\mathcal{M}$ deliver the value $b$ for $\chi$, and structures in the yellow partition $\color{yellow}\mathcal{M}$ deliver the value $c$ for $\chi$. The class of all probability distributions (of observable variables) $\mathcal{P}$ is partitioned into the red partition $\color{red}\mathcal{P}$, into the blue partition $\color{blue}\mathcal{P}$, into the yellow partition $\color{yellow}\mathcal{P}$ and into the grey partition $\color{gray}\mathcal{P}$. Probability distributions in a colour of $\mathcal{P}$ are generated by (potentially many) structures in the same colour of $\mathcal{M}$; the exception is probability distributions in $\color{gray}\mathcal{P}$ which are generated by (potentially many) structures in $\color{red}\mathcal{M}$ and in $\color{yellow}\mathcal{M}$. $P$ is a probability distribution in $\mathcal{P}$ with probability distributions defined similarly for each colour in $\mathcal{P}$.}
\end{subfigure}
\begin{subfigure}{0.8\textwidth}
  \centering
  \includegraphics[width=\textwidth]{U:/GitHub/Covariatesets/Diagrams/Partial.pdf}
  \caption{That probability distributions in $\color{gray}\mathcal{P}$ are generated by structures in $\color{red}\mathcal{M}$ and in $\color{yellow}\mathcal{M}$ creates a complication; the value of $\chi$ is not constant across observationally equivalent structures that $\mathcal{M}$ admits and that generate a probability distribution in $\color{gray}\mathcal{P}$. So $\mathcal{M}$ does not uniformly identify $\chi$. Consideration of the identifying correspondence $G$ determines that this corresponds to there being structures in $\mathcal{M}$ for which $G$ does not deliver the value of $\chi$ when applied to the probability distributions that these structures generate. Nonetheless, if $\mathcal{M}$ restricts the set of values of $\chi$ for any probability distribution in $\mathcal{P}$ then $\mathcal{M}$ does have some non-trivial identifying power for $\chi$. Then $\mathcal{M}$ is said to uniformly partially identify $\chi$ if $\mathcal{M}$ and $\mathcal{P}$ can each be partitioned into countably many disjoint subsets and that a probability distribution in a partition of $\mathcal{P}$ is not generated by a structure in at least one partition of $\mathcal{M}$, holding for any such partition of $\mathcal{P}$. In the context of Figure~\ref{fig:partials} $\color{red}\mathcal{M}$ identifies $\chi$ up to $\lbrace a,c\rbrace$, $\color{blue}\mathcal{M}$ identifies $\chi$ uniquely to $b$, and $\color{yellow}\mathcal{M}$ identifies $\chi$ up to $\lbrace a,c\rbrace$. Each partition of $\mathcal{P}$ includes probability distributions that are generated by structures in at least one partition of $\mathcal{M}$. Equivalently, if $G$ is permitted to be a multivalued functional (or one-to-many) then $\mathcal{M}$ uniformly partially identifies $\chi$ if $G$ exists and if $G(P)$ contains the set of values of $\chi$ that are delivered by structures that generate $P$, holding for all such $P$. A caveat must be applied here; $G$ cannot be trivial in the sense that it is constant across all such $P$. Clearly this definition of $G$ does not exclude the possibility that there is multiplicity of identifying correspondences that satisfy this property. Sharpness is a desirable property in such circumstances; a functional $G$ that can be shown to deliver smaller sets according to some well-defined distance measure across all possible $P$ (and that satisfies the properties above) should be preferred to any alternative identifying correspondence.}
  \label{fig:partial1}
  \end{subfigure}
  \caption{Partial identification of a structural characteristic.}
  \label{fig:partials}
\end{figure}
%==================================================%
%% Bibliography.
\newpage
\bibliographystyle{chicago}
\bibliography{\details Bibliography}
\end{document}